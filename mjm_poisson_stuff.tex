

\section{Some Multi-photon Basics }

While the goal of lithography is high resolution low noise pattern
transfer, the initial analysis of the multi-photon processes
concerns uniform illumination to produce a binary outcome.
First, the probability of success or failure, in essence development
or non-development of a given image feature, is examined as a function
of the expected number of photons doses on that feature using
just the Poisson distribtion. Then, some candidate chemical
schemes are analyzed for their response curves and how they relate
to imaging performance.  Overall the goal is to find a 
system that generates high contrast with a minimum amount of light
while providing low noise patterns and ideally having other features
such as safe and cheap chemical implementations.   
With high resolution patterning, even in the EUV, diffration or other
optical issues will create edge profiles that deviate from perfect
step functions. The part of this illumination profile in the 
ambiguous region of the transfer curve, where success or failure
is not assured, will create noise or uncertain results. So one 
goal is to make the characteristic or exposure curve as step-like
or sigmoidal as limiting by realizable chemical systems and light budget. 

\subsection{ Probability of success when n-photons needed }


The starting point for probability of success or failure when n
indepenent events are required for success is the Poisson Distribution
which is introduced in many optics and statistics textbooks.

\mjmtol{ While not likely to be a source of confusion for this audience, its intersting to note the Poisson Effect may contributing to the competing approach
of imprint lithography 
\cite{Kim_Kyeong_Kwak_Poisson_EffectAssisted_Replication_Lithography_2024}
}

Its worth noting immediately that since we seek a solution and not
just an exercise, that deviations from independene routinely occur
due to the nature of a light source or electrons in the case
of photon and electron counting
\cite{Saito_Endo_Kodama_Electron_counting_theory_1992}
\cite{Ban_Theory_electron_counting_1994}
 both of which may be relevant in formulating a good solution for high-contrast imaging.
Interesting phenomena occur, for example,  in lasers
\cite{Teich_Saleh_Photon_Bunching_1988}
and photolelectron statistics
\cite{Meng_Poisson_photoelectron_statistics_1995}
among others. 

In this case we will consider the probability of getting n or
fewer photons ( or electrons or whatever) as a function of  an expected number
 $<n>$. 
\newcommand{\mjmexpn}{<n>}
\newcommand{\mjmpleqnt}{P(n<=n_t; <n>) }
The probability of getting $n_t$ or fewer things with $\mjmexpn$ is
%\mjmeqn{P(n<=n_t; <N>) = \sum_{k=0}^{n_t} \frac{\mjmexpn^k}{k!} exp(-\mjmexpn)}

\mjmeqn{\mjmpleqnt = \sum_{k=0}^{n_t} \frac{\mjmexpn^k}{k!} exp(-\mjmexpn) \label{eqn:basicsum}}


\mjmtol{ its interesting to note, surprising but of unknown significance,  in passing that there is always a higher probability of getting an even number of photons 

\mjmeqn{P_{even}=\sum \frac{\mjmexpn^{2k})}{(2k)!} exp(-\mjmexpn ) 
=cosh(\mjmexpn)exp(-\mjmexpn )=\frac{1+exp(-2\mjmexpn)}{2}} 

although the opposite is the case of zero is excluded.  
} 

This simple sum  \mjmrefeqn{basicsum} is  a truncated taylor series 
for exp(x) timex exp(-x).

These sums will be used as non-exposure curves versus light dose
or $\mjmexpn$ ( interchangeably with their complementary exposure curves). 
In lithography a binary high-constrast image is
desired which is opposite of what may be desired in other imaging
applications where linearlity and high dynamic range matter. 
Getting a sigmoidal rather than a linear curve early
in the processing may have some benefits over thresholding later
with a deverloper process. 

Important features include the number of zero derivatives at the origin 
and range
of "confusion" where sucess probability is not near zero or one. 
The general form of this sum is
\mjmeqn{ S=P(x)exp(-x) } and its easy
to show that 
\mjmeqn{ \frac{d^mS}{dx^m}=exp(-x)(D-1)^m P(x)  } 
where D is the derivative operator . More specialized to the
current situation, the first 2 derivatives are

\mjmeqn{\frac{d\mjmpleqnt}{d\mjmexpn} = -\frac{\mjmexpn^{n_t}}{n_t!} exp(-\mjmexpn)}

and

\mjmeqn{\frac{d^2\mjmpleqnt}{d\mjmexpn^2} = -\frac{\mjmexpn^{(n_t-1)}}{(n_t-1)!}\left(1-\frac{\mjmexpn}{n_t}  \right) exp(-\mjmexpn)}
with the latter going to zero at $n_t=\mjmexpn $ 
giving the maximum derivative as


\mjmeqn{max(-\frac{d\mjmpleqnt}{d\mjmexpn})  = \frac{n_t^{n_t}}{n_t!} exp(-n_t)
\label{eqn:maxder}}

which brings to mind Stirling's approximation,
\mjmeqn{n_t ! \approx \sqrt{ 2 \pi n_t } {n_t}^{n_t} e^{-n_t} } 

\mjmeqn{max(-\frac{d\mjmpleqnt}{d\mjmexpn})  \approx
 \frac{1}{\sqrt{2 \pi n_t} } \label{eqn:maxderap}  } 


$D_{max}$ seems to decrease with $n_t$, 

\mjmtolx{ this may not be right need to check}

\mjmeqn{\frac{max(-d_{n+1})}{max(-d_n)}= exp(-1)\left(\frac{n+1}{n}\right)^n \approx .368 \left(\frac{n+1}{n}\right)^n }
noting that 
\mjmeqn{ lim_{n\rightarrow\infty} \left(\frac{n+1}{n}\right)^n =e }
this goes to 1 for large n.  
Similarly using \mjmrefeqn{maxderap},

\mjmeqn{\frac{max(-d_{n+1})}{max(-d_n)} \approx \frac{\sqrt{n_t}}{\sqrt{n_t+1}}  }
