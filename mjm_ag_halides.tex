
\section{Silver Halides}

Despite their long history of usage in applications ranging 
from consumer imaging to X-ray and astronomy applications,
 a lot remains unknown about the
details of latent or developed image formation in silver halide films.
A 1980 review did suggest the in real films it took about 4 photons to
create an image by a process involving atom motion \cite[p. 229]{Bose1980} .
Several other short reviews exist highlighting other aspects
such as grain properties and amplification by up to $10^9$
\cite{Tan_Silver_Halides_Photography_1989}.


A turn of the century conference paper reviewed the history and basis
for 1,2, and 4 photon mechanisms \cite{DBLP:conf/pics/Leubner99} with
the stated interest of optimizing a one photon process due to efficiency.
\mjmtol{ The author mentions Tani who did suggest 2 photon processes
suppress dark current. Most solid-state electronic imagers are one photon and have good efficncy as well as linearity but also huge dark current which may be a consequence of this mechanism although need to check the numbers }
The author outlines a system similar to the cascade above with
each photon ( equivalent to an electron here ) adding to a growing
neutral silver cluster,

\newcommand{\mjmcee}[2]{{\centering { \cee{#1 \label{#2} }}}}

\mjmcee{  Ag_n + Ag^+ + e^-  -> Ag_{n+1} }{eqn:halides}
%\cee{  X_n + h$\nu$ -> Ag_{n+1} }

n=4 is thought to be stable and developable while smaller clusters
can decay or fail to develop.  The net equation is then

\cee{  4Ag^+ + 4e^-  -> Ag_{4} }


A two photon process is also described where photo generated holes
are converted to electrons but this seems to rely on existing
$Ag_2$ clusters that either already exist or were photogenerated earlier,

\cee{  2Ag_2(H) +  2h^+  -> 4Ag_{4} + 2e^- }

or

\cee{  2Ag^+ + Ag_2 + 2e^-  -> Ag_{4} }

In considering a one photon process, the author tries to distinguish
optical and thermal processes, " This one-photon process thus needs 
light exposure
to provide both the electron and the hole and is not triggered 
by thermal electron or hole events."
Although no idea what this means. In any case, a one photon process
was considered using "reduction sensitization centers" or similar
electron traps. These mechanisms reduce to
\mjmcee{ h\nu + 2Ag_2 ->Ag_4}
or with the use of Au,
\mjmcee{AuAg_3^+ + e^- -> AuAg_3 }
The former equation is thought to have a photon threshold around 1.4eV
which the author notes is well above thermal voltages suggesting no
dark current but this misses important issues. In any case it is many
thermal voltages ( typically taken as 26meV not the 30 given here ) 
above the silicon band gap and band-to-band generation may be less
for similar quantum efficincies.\mjmtol{ "do the math" somewhere } 




\subsection{Notable featues Silver Halides} 

Superionic conducitivity in AgI 
\cite{Carvalho_Negi_Neto_Direct_calculation_2022} 
which apparently is equivalent to the  
 existence of a Hall Mobility
\cite{Liou_Hudson_Wonnell_Ionic_Hall_effect_1990}
. It appears to persist
in compounds such as $C_5H_6NAg_5I_6$ \cite{Newman_Frank_Matlack_ionic_hall_effect_1977}.


\subsection{Silver Bromide solid} 





Copper sulfides can form varying large unit cell crystals
with correrpsondingly high integer stoichiometry.
They apparently are not molecular crystals.
Unit cells may contain 24 or 62 distinct copper atoms and some phases
contain a sulphur HCP structure with interstial Cu that become
fluid above 100C \cite{Evans_crystal_structures_}.






