\subsection{Some EUV Resists - CAR and MTR etc}

\subsubsection{Chemically Amplified  }

Several good reviews exist
\cite{Ito_Chemical_Amplification_Resists_2005}.
The important points which can be found detailed in most
references relate to multiple chemical reactions in response to
one photon. 
As usually used in the field, chemically amplified resists
sense UV light with a photoacid generator(PAG) and use the
generated acid to catalytically remove groups from a 
polymer. Polymers with enough removed groups are lost
during development.  PAG's appear to commonly use sulfur species.
The "deprotection" scheme is common although polymerization and
depolymerization are also possible. 

A 2022 review compared some CAR's, MOR's, and MTR with an EUV interference
grating illumination pattern demonstrating some benefits of the MTR 
method although all resits had some limitations
\cite{DouglasGuerreroadditionalGillesRAmblard_lithography_resist_2023}.
Interfernce between plane waves would generate a sinusidal intensity
profile and the resulting intensity profile is claimed as
sinusoidal. In any case the line
edges will not be digital or "sharp" and have some profile 
although in actual lithography this may be different and variable
the resist with a high-contrast transfer function may have 
noise benefits. The spatial frequencies or other parameters of the
noise may not be evident and problems like diffusion will not
be considered but sharp transitions create a better starting point
than exponential or gradual ones. 

Attempts to optimize resist parameters for EUV have not always
produced the expected results. A 2018 attempt to add proprietary Mg
and related salts as sensitizers actually decreased EUV absorption
and largely increased roughness while increasing acid generation
and apparent electron generation
\cite{Vesters_Jiang_Yamamoto_Sensitizers_extreme_ultraviolet_2018}.
The ability to change the electron-hole pair generation energy,
usually estimated using Klein's 3x the bandgrap rule in semoconductors,
creates some interesting possibilities although impoved photon
efficiency would probably help quanizatin problems more.

More recently around the turn of the century, some benefits were
observed with Sn in the right environment without amplification
\cite{Belmonte_Cendron_Reddy_Mechanistic_insights_2020}.


In the prior curves, "photon" was used as a generic term
but in reality reactions could be initiated by electrons
fron an initial absorption event or from the ambient.
A background of ions may exist which could create a uniform
fog or exposure and a multi-event resist may be able to
avoid some of this pattern deterioration. 





\subsubsection{Multi-trigger }

Multi-trigger resists appear to be the closest thing to
the multi-photon idea applied to DUV or EUV lithigography.
A 2018 paper describes the concept as spec
"The multi trigger
resist (MTR) is a negative tone crosslinking resist that does not need a post exposure bake
(PEB)"
but gets to multi event property
"e reaction will only proceed where
an MTR molecule and a crosslinker are
simultaneously activated in close proximity to each
other"
\cite{Popescu_Vesters_McClelland_Multi_Trigger_Resist_2018}
although this work does not appear to show any rate equations or
characteristic curves.

\mjmtol{ make sure to come back after doing CAR as the acid 
dynamics are unclear among other possible problems with the words.}

A later paper 
\cite{RoelGronheidadditionalDanielPSanders_Multi_trigger_resist_novel_2019}
described some resist design considerations, "
resolution, line edge roughness and sensitivity requirements, 
with minimal defectivity. However, these parameters
are linked by a fundamental trade-off in lithography (the RLS triangle)
and it is difficult to overcome. For instance, addition of quenchers in
chemically amplified resists (CAR) reduces the acid diffusion length and
increases the resolution of the patterned features, but decreases the sensitivity,
and impacts on material stochastics affecting the line edge roughness. Defectivity
due to line collapse, bridging or line breaks is also a fundamental problem."
 
The authors go on to illustrate the concept with pictures although no
obvious reactions or rate equations.  From the pictures, the reaction
sequence may be, 

\cee{X + $h\nu$ -> P } 

\cee{A ->[P] $A^\prime$ } 

\cee{B ->[P] $B^\prime$ } 

\cee{$A^\prime +B^\prime$  -> C  } 

With C being the developer selectivity component.
Rate equations would then be,

\mjmeqn{ \mjmdt{X}= -\phi X }

\mjmeqn{ \mjmdt{P}= -\mjmdt{X} }

\mjmeqn{ \mjmdt{A}= -  A P ;  \mjmdt{A^\prime}= A P - \mjmdt{C} }

\mjmeqn{ \mjmdt{B}= -  B P ; \mjmdt{B^\prime}= B P- \mjmdt{C} }

\mjmeqn{ \mjmdt{C}= A^\prime B^\prime }

X and P are straightforward,

\mjmeqn{ X= X_0 exp(-\phi t) }
\mjmeqn{ P= X_0(1- exp(-\phi t)) }
with A and B being identical but more complicated,
\mjmeqn{ \mjmdt{A}= -  A X_0(1-exp(-\phi t))   }
\mjmeqn{ -ln(A) =  X_0(t+\frac{1}{\phi}exp(-\phi t))+c   }
\mjmeqn{ ln(A) =  X_0(-t-\frac{1}{\phi}exp(-\phi t))+c   }

\mjmeqn{ A =  A_0exp(-X_0 (t-\frac{1}{\phi}(1-exp(-\phi t))))   }
as $\phi$ goes to zero this seems to have the right limit,

\mjmeqn{ A \approx  A_0exp(-X_0 (t-\frac{1}{\phi}(\phi t)))   }

Under these conditions A and B are identical leading to

\mjmeqn{ \mjmdt{A^\prime}= A P - (A^\prime)^2 }
\mjmeqn{ \mjmdt{A^\prime}= 
A_0X_0exp(-X_0 (t-\frac{1}{\phi}(1-exp(-\phi t))))(1- exp(-\phi t)) 
- (A^\prime)^2 }

\subsection{Multi-trigger with quencher }

Perhaps a little more accurately, consider the system
with X and Z dead or inactive components, 


\cee{A + $h\nu$ -> A^* } 

\cee{Q + $h\nu$ -> X } 

\cee{A^*  -> A } 

\cee{A^* + Q -> Z } 

with the final polymerization step being left vague as issues
with strnaded monomers and crosslinking are ignored right now, 

\cee{M+P ->[A^*] P } 

\cee{M+M ->[A^*] P } 

Taking the last process first, assume that a monomer can be attached
to anything anywhere and that P regardless of shape or length
is left after development,  

\mjmeqn{ \mjmdt{P}= A^* M  }

\mjmeqn{ \mjmdt{M}= -  A^* M  }

The others are straightforward,

\mjmeqn{ \mjmdt{A}= -A\phi  }
\mjmeqn{ \mjmdt{Q}= -Qk_q\phi  }

\mjmeqn{ \mjmdt{A^*}= A\phi  -\frac{1}{\tau} A^* -Q A^* }


initial amounts of A,Q, and M are present and maybe drop time
constant.  As with other cases,
\mjmeqn{ A(t) = A_0exp(-\phi t)  }
\mjmeqn{ Q(t) = Q_0exp(-k_q\phi t)  }

\mjmpicture{quench.jpg}{ The quencher is paramterized to be destroyed before the catalyst is activated and destriryed. The initial slope should be zero but if the catalyst is depleted the final amount of polymer is limited below 1.  $  ./mjm\_poisson.out "test test=mtrigq;a0=5;tau=10;phif=10;q0=100;t=1;kq=20;dt=.001" quit  $ }{quench}




\mjmeqn{ \mjmdt{A^*}= \phi A_0exp(-\phi t)
				-Q_0exp(-\phi t) A^*   -\frac{1}{\tau} A^*  }
Using an integrating factor,
\mjmeqn{ F = (\int ( Q_0exp(-\phi t)+\frac{1}{\tau})dt)   }
\mjmeqn{ F = (-\frac{ Q_0exp(-\phi t)}{\phi}+\frac{t}{\tau})  }

\mjmeqn{A^* = exp(-F)\int(exp(F)A_0exp(-\phi t) dt) + C exp(-F) }
The integral may be doable,

\mjmeqn{I = \int(exp( -\frac{ Q_0exp(-\phi t)}{\phi}+\frac{t}{\tau}) A_0exp(-\phi t) dt) }


\mjmeqn{I_t = \int(exp( -\frac{ Q_0exp(-\phi t)}{\phi}+\frac{t}{\tau})
(  A_0exp(-\phi t)+\frac{A_0}{Q_0\tau}) dt) }

\mjmeqn{\frac{d}{dt} (exp( -\frac{ Q_0exp(-\phi t)}{\phi}+\frac{t}{\tau}))
= (Q_0exp(-\phi t) +\frac{1}{\tau}) exp(F)  
}

\mjmeqn{I_t = \frac{A_0}{Q_0}exp(F)}




