


\section{Early Events and Ultimate Fate }

One important aspect of the lithographic process is
the early steps in converting the photon into
a latent image. Two processes are commonnly considered.
The more common involves creation of a single electron 
which may thermalize much later. Alternatively several 
electrons could be created almost simultaneously from
the same atom.
With the interest in metal oxide resists and the topic
of this work being largely silver halides, details of
absorption in metals such as silver and tin may be useful
to describe and compare where known. 

Silver is the immediate issue but tin is of importance
too. It appears there are a lot of small issues that likely
matter to simple macroscopic results. 
Its worth noting that just recently the potential of the lead
acid battery was explained by adding relativistic terms to 
the pseudopotential for the valence states
\cite{Ahuja_Blomqvist_Larsson_Relativity_Lead_2011}.
The authors contrast lead to tin by the nuclear charge and
then attribute several properties to the additional
relativistic effects. Core electrons differ between
elements placing different contstraints on outer
shell behavior. 


Due to the participation of possibly lower level shells in
relevant transitions, some DFT pseudopotential calculations
may be limited. A variety of specialized techniques exist
that may provide details on initial EUV-resist interaction.
A Lanczos theory applied to inner shell absorption and photoionization
was recently published
\cite{Moitra_Coriani_CabralTenorio_Inner_shell_photoabsorption_2021}.
Variations on the EOM-CCSD method exist such as an
STEOM  or similarity transformed approach \cite{doi:10.1021/acs.jctc.4c01181}.
One problem seems to be dealing well with the continuum free states
which is addressed by several methods
\cite{doi:10.1021/acs.jctc.1c00303}.

Tables of element EUV absorption coefficients have produced
and show several metals including Sn, Sb, Ag, Cs, Bi, Pb have
comparably high molar absorption rates  along with I 
\cite{PMC11433861}. Translation into resist performance of course
is complicated by chemical environment and actual fate of the
absorbed light energy. Paradoxes and surprises have been observed
as mentioned elsewhere. As some resists molecules have dimensions 
approaching the light wavelength especially with a high dielectric constant
various "particle in a box" or antenna effects may be considered.
Quantum effects and relativity probably dominate.  

A study of oxo caged tin and tin butyl clusters below 70 ev
demonsrated a negative tone resist with exposed areas losing
organics \cite{PMC10926160}. The later discussion of Sn-butyl
clusters is vaguely similar to the issues with silver halide
clusters although many specific are different. 

While not directly applicable to photoresist design, a lot 
of work over the last few decades
 has explored the photocathode properties of some 
materials containing high absorption elements.
A 1996 work looked at stability and quantum efficiencies
of materials such as metal halides such as AgCl and CsI
into the 1 nm range
\cite{OswaldHWSiegmundfirstMarkAGummin_title_Progress_1996}.
While there was a significant angle dependence of emission
its not clear how the test geometry influenced this and
if it is relevant at all to photoresist under high NA exposures.
Resists would generally want to minimize electron emission
and instability may indicate latent image formation.
Although if the emitted electron does not cause exposure
of random resist, the charging of the exposed region could
cause a variety of other effects. 




A 2025 work discusses some EUV resist issues and results with silver
and gold dry-develop resists with n-heterocycle complexes
\cite{https://doi.org/10.1002/smll.202407966}. While one gold resist
was ultimately selected for further evaluation, a neglected story
may exist in their figure 2c. In these 3 eposure curves 2 appear
more or less exponential while the third one, with 2 Ag centers
denoted (i-Pr)L-AuCl demonstrated no response below a threshold of 
about 100mj/cm2 followed by a concave downward ( like $e^n, n<1$ ) 
approach to 1. The authors subsequently dismiss this resist as the
PF6 provides better sensitivity. It would be interesting to look at
the initial part of these curves to see if there is an observable
delay period. Simple "sensitivity" issues should not cause
a dealy but rather an exponential over a longer dose range.
\mjmtol{ Also they elaborate on electron thermalization with nice pictures
but AFAICT this is really an unknown and needs to be investigated.  
I became aware of how glib some filler material is from the medial
literature. Someone probably said it and it is plausible but
never verified although the details here could be important. }


As early as 2022 good reviews exists on metal containing
e-beam resists with possible utility for EUV. For example,
Saifullah et al consider self-developing metal oxide and halide
resists and discuss less known mechanisms such as Knotek-Feibelman
which can create 3 electrons on impact
\cite{Saifullah_Tiwale_Ganesan_Review_metal_containing_2022}.
They go through metal sulfoxide resists characteristic curves
exhibiting a threshold charge dose to begin developing a latent image.
Silver is mentioned in passing but not as a halide.

It turns out that lead halides and related perovskites containing
organics have been investigated for photovoltaic performance
that provides some information on EUV issues.
One recent work suggests they have a lot of remarkable
properties  and go on to do some analysis in the EUV region
\cite{Green_Jiang_Soufiani_Optical_Properties_Photovoltaic_2015}.
While the goals of these efforts are to produce stable energy
sources, their susceiptibility to decomposition under illumination
is exactly a desired feature of a photoresist. 
Both lead and iodine have decent EUV absorption and the possibilities
of designing conversion of light into decomposition may be worth
exploring. 

Its also worth keeping in mind while exploring resist candidates that
the resist itself could form the desired features such as conductors.
This process has been explored in a set of techniques called direct
optical lithogrphy\cite{D4TA06618A}. As several resists begin with
metal compounds that become more metallic in exposed areas the
potential is there to directly form condcutors.  

An earlier thesis from 2016 explored an HfO resist which appeared
to use sulfate to control aggregation 
\cite{Luo_Deposition_characterization_patterning_2016}.
The resist was also explored for response to electrons and found
to response to energies as low as 2eV with a high enough dose
although later XPS analysis of sulfur content suggests some details
need to be explored.
A 30kev He+ ion beam was also found to work with 50-100 times
more sensitivity than 30kev electrons. In the reiew of past methods,
this thesis also mentions $Ag_2S$ and similar resists and their
limitations.  




