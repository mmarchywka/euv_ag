



Initial discussion on JDFTX issue forum suggested looking at 
GW/BSE. A quick liteerature search found a few existing and active
projects although not all open. Introductory material
often introduces the Green Function and self energy in the context
of X-ray interactions with matter and several open articles
exist. A 2022 work introduces Green's function with Fermi Golden rule
 progressing to multi-pole and multi-electron with Debye-Waller
vibrations 
results  is compared with X-ray absorption and XPS experiments
\cite{Kas_Rehr_Vila_Green_functions_applied_2022}. 
A 2008 much more detailed paper discusses similar concepts
and the code FEFF 
\cite{Rehr_Kas_Prange_initio_theory_2009}.

A 2024 numeric all electron implementation of GW/BSE used reolution of identity
and implemented in not ao opwn FHI-aims code 
\cite{Zhou_Yao_Blum_electron_2024}.
A 2025 paper used analytical approximations including a multipole-Pade 
expression for the Green's Function
\cite{2025arXiv250109121L}.
A 2006 paper introduced the OCEAN code with results mostly above 200 eV
\cite{Gilmore_Vinson_Shirley_Efficient_implementation_core_2015}.
Another short introduction with good match to experiement was
a 2007 conference paper 
\cite{Rehr_Kas_Prange_Inelastic_Losses_Multi_2007}.

Generally relativistic effects are not mentioned although the Dirac
Equation does come up in passing. Usually the ground state is found
with DFT which of course could use fully relativistic pseudopotetials.
They may be ignored even in heavy elements leading to things like 
difficulty finding potential of lead-acid battery versus tin
\cite{Ahuja_Blomqvist_Larsson_Relativity_Lead_2011}
. As comparing heavier elements will likely be important including
relativity consistently may be worthwhile.
QED has also been included in pseudopotential design with reduced
core states
\cite{Zaitsevskii_Mosyagin_Oleynichenko_Generalized_relativistic_smallcore_pseudopotentials_2023}.
Fully relativistic BSE and DCB ( Dirac Coulomb Breit ) are current
topics \cite{Ferenc_Matyus_Born_Oppenheimer_Dirac_2023}.



Its also likely that results from surface-light interactions and
2D materials will be thought provoking or useful. Surface enhanced
interactions with candidate elements such as Ag are well known.
Literature here is diverse and may include things like surface
enhanced Raman or topological insulators. 

Simplifications can rely on approximations or a shift from numerical
to more analytical techniques. There is some hope that a more complete
physical dscription, fully relativistic with retardation, could
lead to some easier analysis but that remains to be demonstrated.



